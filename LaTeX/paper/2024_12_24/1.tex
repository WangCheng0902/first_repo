\documentclass[12pt]{article}

% 导入中文支持的包
\usepackage[UTF8]{ctex} % ctex 包支持中文处理

% 设置页面布局(可选)
\usepackage[a4paper,margin=1in]{geometry}

% 导入常用包(根据需要添加)
\usepackage{amsmath, amssymb} % 数学公式
\usepackage{graphicx} % 插入图片
\usepackage{hyperref} % 超链接
\usepackage{xcolor} % 颜色支持

% 标题和作者信息(可选)
\title{颅内压生理学和脑脊液动力学的仿体模型}
\author{王成}
\date{\today}

\begin{document}

% 标题
\maketitle

% 目录(可选)
\tableofcontents
\newpage

% 正文
\begin{center}
    \section*{摘要}
\end{center}

我们在此描述了一种新颖的与颅腔等大的仿体模型,并给出了其验证过程。包括脑室、池区和蛛网膜下腔在内的脑脊液区域由核磁共振成像获得。
脑的机械特性和颅脊顺应性是基于已发表的数据设定的。实现了对脑脊液流动进行整体和脉动的生理建模。通过将模型中的流量和压力测量结果与健康受试者的体内数据进行比较来验证模型有效性。
脑室内记录的生理颅内压平均为10 mmHg,脉冲峰值幅度为0.4 mmHg。在大脑导水管和蛛网膜下腔中分别测得脑脊液流速峰值为0.2和2 ml/s。
本模型是一种首次尝试在体外模拟生理颅内动力学的方法。在此,我们描述了模型的设计和制造,操作参数的定义和实现,以及所模拟动力学的验证。

索引术语——解剖模型,顺应性,蛛网膜下腔(SAS),脑室系统。

\section*{\uppercase\expandafter{\romannumeral1}.引言}

脑脊液(CSF)有助于中枢神经系统(CNS)的体内平衡。在颅腔中,由于受到大气压强的作用,脑脊液被局限在脑室和蛛网膜下腔(SASs)中。
脑脊液通过浮力支撑大脑,保护其免受冲击,输送营养物质和神经内分泌物质,并清除代谢废物。

CSF动力学的变化与多种疾病有关。例如,脑积水和脊髓空洞症已被证明与CSF总体流动及脉动的紊乱有关。然而,这些关系的具体细节尚不清楚。

颅内动力学模型可以提高对中枢神经系统病理生理学的理解。集总参数和计算流体动力学(CFD)模型均已用于表征CSF空间以及主要颅内动脉的动力学特征。
集总参数方法特别适合用于颅内动力学的整体经验描述。相应的模型参数,如脑脊液流出阻力和压力-体积指数,在过去的十年中已成为临床实践中的标准。
另一方面,CFD模型可以用于获取通过测量无法获得的空间分辨流动信息。

然而,无论是集总参数模型还是CFD模型都未被证明是开发及优化颅腔动力学相关医疗器械的理想选择。
例如,用于治疗脑积水的脑脊液分流器根据ISO 7197标准通过实验方式进行测试以评估液压阻力。
然而,除了伦理问题,比起小鼠、老鼠或兔子,尤其是当使用诸如狗、山羊或猴子等脑内动力学与人类更为接近的大型动物时,实验成本将会非常昂贵。
体外模型代表了第四种可能有助于研究中枢神经系统病理生理学的模型类型。解剖细致和简化的模型都已被用于验证MRI序列、脑力学的计算模型,以及第三脑室中脑脊液流动。在脑损伤研究中,模型被用于研究组织对冲击的反应。
据我们所知,仅有一个关于脊柱脑脊液空间中流体和压力动态的模型被报道过;该模型被用于理解脊髓空洞症。

颅内空间的仿体模型有可能减少、改进,并在较小程度上取代动物模型用于分流器和其他神经外科器械的测试。
实现此类应用的一个重要步骤是复制健康状态下的颅内动力学。我们在此展示了首创的颅内腔体仿体模型,该模型能够再现生理性的脑脊液和压力动态。
我们报告了幻影的设计、开发及与文献中描述的体内数据的验证,显示这种建模方法可以促进对颅内动力学的理解。

\section*{\uppercase\expandafter{\romannumeral2}.材料与方法}

\subsection*{A. 脑脊液和脑室系统}

\section*{\uppercase\expandafter{\romannumeral3}.结果}

\section*{\uppercase\expandafter{\romannumeral4}.讨论}




\end{document}
