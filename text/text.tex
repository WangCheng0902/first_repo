\documentclass[titlepage,12pt]{article}
\usepackage[UTF8]{ctex}
\usepackage{xeCJK}
\usepackage{graphicx}
\usepackage{lettrine}
\usepackage{shapepar}
\usepackage{enumitem}
\usepackage{listings}
\usepackage{geometry}
\geometry{a4paper,left=3cm,right=3cm}

\title{Questions and Notes}
\author{wang cheng}
\date{Dec 2024}

\begin{document}

\maketitle

\section{笔记}
\subsection{Intracranial pressure and  cerebral blood flow}
\subsubsection{摘要}
颅内压由脑组织、脑脊液和脑血容量三者的体积决定,任一内容物的变化都需要其他内容物来补偿变化内容物的体积,
否则颅内压会发生变化。颅内压的相对恒定对于维持脑灌注压处于正常是至关重要的,正常的脑灌注压使得脑血流在全局和局部得到调节,
以防止系统性动脉血压发生变化引起的脑灌注过度或不足,并满足不同脑区域动态氧气和底物需求。
\subsubsection{脑组织}
脑实质大约有1400克,由神经元、胶质细胞和细胞外液组成。血脑屏障(BBB)由毛细血管内皮细胞之间的紧密连接构成,
将血液与脑间质液分隔开来,以为神经元活动提供适宜的环境。大脑组织病理性增加的原因包括肿瘤、细胞毒性水肿(由于细胞膜破裂)和血管源性水肿(由于BBB破坏)。
\subsubsection{脑脊液}
脑脊液位于蛛网膜和软脑膜之间。其功能包括维持稳定的化学环境,支持代谢产物和神经传递物质的运输,并提供葡萄糖。可以说大脑漂浮在脑脊液中,
脑受线性和剪切机械力的影响也通过脑脊液的保护性位移而减少。
\subsubsection{脑血容量}
脑的动脉血供是由颈内动脉和椎动脉,分别通过颈动脉管和枕大孔进入颅内。血液通过脑静脉、静脉窦和颈内静脉排出,通过颈静脉孔从颅腔排出。
正常颅内血容量约为150毫升,其中三分之二位于静脉系统中。


\end{document}
